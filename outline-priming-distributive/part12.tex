

\begin{document}

\subsection{Experimental literature and the covariation issue}

The non-distributive/distributive distinction has been extensively explored in the psycholinguistic literature, from both processing and developmental perspectives (CITE). However, while studies might differ in the specific experimental approach, they have all been exclusively focused on ambiguities arising from mixed transitive predicates (CITE). 

%Experimental approaches to the non-distributive/distributive distinction have been almost exclusively focused on ambiguities arising from mixed transitive predicates (CITE). 

As observed, distributive readings of transitive predicates such as \ref{ex.transitive.mixed.paint.LF}b can only be fully isolated in scenarios where there is covariation of objects (i.e. these situations are the only ones that make distributive readings true and non-distributive readings false). In consequence, most experimental attempts to investigate the availability of distributive inferences have tested these sentences in scenarios such as the one described in \ref{ex.transitive.mixed.paint}b (CITE). 

This approach, however, has a main drawback: The use of covariation scenarios might cause people to develop verification strategies, which are not inherent to distributivity but to its specific instantiation. For instance, in order to verify whether the distributive reading of \ref{ex.transitive.mixed} is true or false, one could directly check whether there is more than one sand castle: having a ``number mismatch'' between the sentence and the number of objects in the scenario would be enough to accept the sentence. As a result, the processing pattern that many studies had attributed to distributive interpretations might be in fact characteristic of covariation effects and not of distributivity per se (i.e. distributive entailment). 

Specially relevant for our purposes is the covariation issue in  Maldonado et al priming study. 
Maldonado et al (2017) tested whether the distributive/non-distributive contrast for sentences such as `Two squares are connected to three circles'' could give rise to priming effects. 
Participants were asked to performed a sentence-picture matching task, where they had to choose between two images the one that best described the sentence. Crucially, distributive readings were always instantiated by a ``covariation'' picture, whereas non-distributive readings were instantiated by a non-covariation picture (e.g. \ref{fig:example.maldonado2017}). 
In the so-called \emph{prime} trials, only one of the images was compatible with the sentence under one of its possible readings, whereas in \emph{target} trials, both images were compatible with the sentence under each of its readings. 

When subjects were biased towards one of the two readings in one prime trial, they were more likely to access the same interpretation in a subsequent target trial. This effect was asymmetric: only distributive primes had a significant impact in the subsequent target, whereas non-distributive primes were not  significantly different from having no prime at all.  

The authors concluded that priming of semantic interpretations was at play. Given the asymmetry of the effect, their results were considered to be compatible with the idea that distributive readings are derived, and what we primed was the insertion of the $D$ operator.

- See the specific case where distributive readings can have assoiciated a non-covariation scenario. 

- This supposes a potential confound for the interpretation of Maldonado et al results.

- The priming effect attested in Maldonado et al 2017 might be partially explained as a verification strategy priming. 

- When participants are forced to access a distributive interpretation (distributive primes), a verification strategy consisting on checking covariation of the objects named in the sentence might be primed, giving rise to the attested priming effect. 

%%%


- The same confound arises for most experimental studies, which use covariation scenarios to isolate distributive readings. The processing pattern that many studies have attributed to distributive interpretations might be in fact characteristic of... 

- As observed, this confound cannot be easily avoided in the case of transitive mixed predicates. Alternatively, one could start looking at the case distributivity in adjectival mixed predicates, which do not involve covariation effects. 
\end{document}


