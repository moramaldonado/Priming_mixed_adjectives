\documentclass[a4paper]{article}
\begin{document}
Phenomena

The sentences in 1 and 2 can have at least two different readings: a distributive reading illustrated in X and a collective or non distributive interpretation in b. 

- Ejemplo Transitive

- Ejemplo Adjectivo 


- Distributivity can be diagnosed by the existence of a distributive entailment such that , when applied to a plural subject, it is inferred to be separately true of each individual member of that subject

- is considered collective if, when applied to a plural subject, it is not inferred to be separately true of each individual member of that subject, but only of the subject as a whole

- The availability of both collective and distributive interpretations has been linked to mixed predicates. 

- Note that not all predicates are mixed. Some are just inherently distributive in that the distributive entailment is obligatory for them; or inherently collective. 

- This is both the case for transitive and adjectival predicates. 

- Mention stubbornly distributive predicates. 

- A difference has been traditionally made between P and Q distributivity. ?? 
Whenever the plurality smiles, so do its members. 
Using the same approach for 

- The way out of this problem is to introduce a covert distributive operator in the logical representation that can induce covariation of indefinites and binding of pronouns. ?is is the purpose of the D operator postulated by Link (1987, 1991) and Roberts (1987). ?is operator shiſts a verb phrase to a distributive interpretation, more specifically, one that holds of any entity X each of whose singular individuals satisfy the un-shiſted verb phrase.
introduces a universal quantiifer

% binding pronouns

- Differences between these two types of predicates in covariation and isolation. 

Distributivity and covariation

- As observed in X, when the D operator is applied to a transitive predicate, it not only guarantees the distributive entailment but also allows for covariation. covariation of indefinites and binding of pronoun

- Covariation is not inherent to distributivity, even for transitive predicates. 
EXAMPLE WITH PAINT

- Which are the readings obtained by inverse scope of Distributive readings. Potential confound. 

- Transitive predicates where the distributive entailment is mandatory, do they still need the D operator? If not, in these cases we could have the distributive operator only when we want the indefinite to covary. It has been said that ``The difference between lexical and phrasal distributivity corresponds to the difference between what can and what cannot be ascribed to the lexical semantics of the verb.'' Now, do you think in the case of ``see something'' 

- Scope ? Note that there are some readings that arise as a product of the combination between scope and distributivity


\end{document}